%%%%Select the class of document
\documentclass{amsart} %type/class of article
%\documentclass{article}
%\documentclass{book}
%\documentclass{letter}

%%%%Select packages
\usepackage{amsthm,amsmath,amssymb} %math packages (always include)
\usepackage{geometry} %can be used to modify page dimensions, etc.
\usepackage{graphicx} %figure
\usepackage{float} %figure position in pdf
\usepackage{multirow} %table with multirow
\usepackage[labelfont=rm]{subcaption} %subcaption of figure
\usepackage[numbers]{natbib} %management of bibliography
\usepackage{listings} %use to list command or else in block
\usepackage{xcolor}  %allow to display color
\usepackage{url}
%%%%% End Select packages


%%%% Define a style used to display a block of Latex code
\lstdefinestyle{TexStyle}{
language={[LaTeX]TeX},
frame=single,
backgroundcolor=\color{white},
basicstyle=\small\ttfamily,
morekeywords={maketitle,includegraphics},
keywordstyle=\color{blue},  
commentstyle=\color{gray},
stringstyle=\color{black}
}


%%%% Set up title page information
\title{CAAM 520: Computational Science II \\
Homework 1.}
\author{Wei Wu}
%\date{\today}
%%%% End Set up title page information


%%% Begin document
\begin{document}

%%%% Write an abstract if needed
%\begin{abstract}
%You can add an abstract at the beginning of the article.
%\end{abstract}

%%%% Make title page
\maketitle

\section{Introduction} 
In this project, via finite difference method we solved a 2D Laplace equation $-(\frac{\partial^2 u}{\partial x^2} + \frac{\partial^2 u}{\partial y^2}) = sin(\pi x)sin(\pi y)$ with Dirichlet boundary conditions $u(x,y) = 0$. We solved the problem on a $(N + 2) \times (N + 2)$ meshgrid on $[-1,1]^2$, with $h = 2/(N+1)$. We implemented a solver via Weighted Jacobi method for the resulting linear system. The Weighted Jacobo method is referenced from here:  \url{https://en.wikipedia.org/wiki/Jacobi_method}. 

Note that by assuming zero boundary conditions, the problem is reduced to solving an $N \times N$ system for the interior nodes.   


\section{Testing and verification}
For a small $N = 2$ grid, we could hand compute the solution by solving the following system:

\begin{gather}
\frac{3^2}{2^2}
\begin{bmatrix} 4 & -1 & -1 & 0 \\ -1 & 4 & 0 & -1\\
				-1 & 0 & 4 & -1  \\ 0 & -1 & -1 & 4
 \end{bmatrix}
 \begin{bmatrix} u_5 \\ u_6 \\ u_9 \\ u_{10} \end{bmatrix}
=
\begin{bmatrix}
0.75 \\ -0.75 \\ -0.75 \\ 0.75
\end{bmatrix}
\end{gather}

This gives us the solution 

\begin{gather}
\begin{bmatrix} u_5 \\ u_6 \\ u_9 \\ u_{10} \end{bmatrix}
=
\begin{bmatrix}
0.0556 \\ -0.0556 \\ -0.0556 \\ 0.0556
\end{bmatrix}
\end{gather}

This corresponds to the solution produced by our solver. See u\textunderscore N=2.dat for the produced solution. 

\section{Results}
For a residual $< 1e-6$. 

When N = 10, the max error is 0.001372. It takes 524 iterations and about $7 \times 524$ many operations to converge. Wall time = 0.005255 seconds.

When N = 20, the max error is 0.000379. It takes 1818 iterations and about $7 \times 1818$ many operations to converge. Wall time = 0.033477 seconds

When N = 30, the max error is 0.000175. It takes 3845 iterations and about $7 \times 3845$ many operations to converge. Wall time = 0.109254 seconds


\section{Scaling}
Since I stored all of $u, u_{true}, b$ as static arrays of doubles, it takes about $3 \times (N+2)^2 \times 8$ bytes of memory.  

For N = 10, it takes 0.005255 seconds for the solver to converge. Estimated memory usage is 384 bytes.

For N = 100, it takes 9.026760 seconds for the solver to converge. Estimated memory usage is about 0.2496 Megabytes.

When I tried N = 1000, the program gave me segmentation fault. This is because the memory usage is about $3 \times 8$ Megabytes, which is more than C's stack size. The limiting factor is hence the memory.

I then tried N = 500. It takes 5009.520565 seconds with 791297 many iterations. The limiting factor is hence run time.

My conclusion is my code can run reasonably well for N equals to a couple hundreds.   


%%%% End document
\end{document}
